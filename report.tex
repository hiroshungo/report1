\documentclass[a4paper,12pt]{jsreport}
\usepackage[utf8]{inputenc} 
\usepackage[dvipdfmx]{graphicx}
\usepackage{url}

\title{レポート課題1}
\author{}
\date{2025年5月8日}

\begin{document}

\maketitle

\section{課題内容}
自分の手で「mv」,「cp」,「cp -r」の挙動を調査して,そのレポートをOverleafで作成してください.

\section{方法}
「mv」は\cite{bib:mv},「cp」は\cite{bib:cp},「cp -r」は\cite{bib:cp-r}を参考に,各コマンドを実行する.

\section{結果}
 a.txt,b.txt(テキストファイル)および folder1,folder2(フォルダ)を対象に,「mv」,「cp」,「cp -r」コマンドを実行した際の挙動の比較を表~\ref{tbl:lcr} に示す.\par また,これらのコマンドにより生じた一連のファイル操作の結果を,図\ref{fig:chart1},図\ref{fig:chart2},図\ref{fig:chart3}に示す.

% 表
\begin{table}[h]
\caption{mv, cp, cp -r の挙動比較}
\label{tbl:lcr} 
\begin{tabular}{|l||c|}
\hline
\textbf{コマンド} & \textbf{挙動} \\
\hline
\texttt{mv a.txt [ディレクトリ名]} & \texttt{a.txtが[ディレクトリ名]に移動した} \\
\hline
\texttt{cp a.txt b.txt} & \texttt{a.txtがb.txtとして複製された} \\
\hline
\texttt{cp -r folder1 folder2} & \texttt{folder1とその中身がfolder2にコピーした} \\
\hline
\end{tabular}
\end{table}

% 図

\begin{figure}[h]
\centering
\includegraphics[width=0.9\textwidth]{chart1.pdf}
\caption{「mv」の一連のファイル操作}
\label{fig:chart1}
\end{figure}

\begin{figure}[h]
\centering
\includegraphics[width=0.9\textwidth]{chart2.pdf}
\caption{「cp」の一連のファイル操作}
\label{fig:chart2}
\end{figure}

\begin{figure}[h]
\centering
\includegraphics[width=0.9\textwidth]{chart3.pdf}
\caption{「cp -r」の一連のファイル操作}
\label{fig:chart3}
\end{figure}

\bibliography{ckf.bib}
\bibliographystyle{junsrt}

\end{document}
